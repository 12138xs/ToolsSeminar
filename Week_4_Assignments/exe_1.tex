% !TeX encoding = UTF-8
% !TeX program = PdfLaTeX

\documentclass[UTF8,11pt]{ctexart}
\usepackage{geometry}
\geometry{scale=0.8}
\usepackage{enumitem} % adjust spacing
\usepackage{graphicx} % figure environment
\usepackage{amsthm} % theorem environment
\usepackage{amsmath} % matrix environment
\usepackage{amsfonts} % bold math symbols
\usepackage{float} % [H]
\usepackage[nottoc]{tocbibind} % add bib to tableofcontents

\title{数学分析与线性代数例题}
\author{佚名}
\date{2019 年 12 月 6 日}

\newtheorem{theorem}{定理}
\newtheorem{sample}{例}
\newtheorem*{solution}{解}

% \linespread{1.5} % line spacing
\begin{document}

\maketitle
\tableofcontents

\section{微分中值定理及其应用}
\begin{theorem}[极值的第二充分条件]
    设 $f(x)$ 在 $(x_0 - \delta , x_0 + \delta)$ 可导且 $f'(x_0)=0$,又 $f''(x_0)$ 存在.
    \begin{itemize}
        \item[1)] 若 $f''(x_0) < 0$ ,则 $f(x_0)$ 是严格极大值;
        \item[2)] 若 $f''(x_0) > 0$ ,则 $f(x_0)$ 是严格极小值.
    \end{itemize}
\end{theorem}

\begin{sample}
    求 $y = \frac{1}{3} x \sqrt[3]{(x - 5)^2}$ 的极值点与极值\footnote{原题摘自《数学分析简明教程》(上册)P142.}.
\end{sample}

\begin{solution}
    函数在 $(-\infty, +\infty)$ 上连续,当 $x \neq 5$ 时有
    \begin{equation}
        \label{eq:jizhi}
        y'= \frac{1}{3} \left( (x-5) ^ \frac{2}{3} + \frac{2x}{3} (x-5) 
        ^ {-\frac{1}{3}} \right) = \frac{5(x-3)} {9(x-5) ^ {1/3}} .
    \end{equation}
    令 $y' = 0$ 得稳定点 $x = 3$,现列表如下:

    \begin{table}[htbp]
        \centering
        \begin{tabular}{|c|c|c|c|c|c|} \hline
            $x$ &  $(-\infty , 3)$ & $3$ & $(3 , 5)$ & $5$ & $(5 , +\infty)$\\ \hline
            $y'$ & $+$ & $0$ & $-$ & 不存在 & $+$\\ \hline
            $y$ & $\nearrow$ & $\sqrt[3]{4}$ & $\searrow$ & $0$ & $\nearrow$\\ \hline
        \end{tabular}
    \end{table}
    从表中可见 $x = 3$ 是极大值点,极大值为 $f(3) = \sqrt[3]{4}$;$x = 5$ 为极小值点,
    极小值为 $f(5) = 0$. 我们可以大致地画出函数的图形,如图1所示.
    
    \newpage
    \begin{figure}[htbp]
        \centering
        \includegraphics[width=8.6cm]{fig/function.pdf}
        \caption{$y = \frac{1}{3} x \sqrt[3]{(x-5)^2}$ 的函数图像}
        \label{fig:function}
    \end{figure}
\end{solution}

\section{行列式}
\begin{sample}
    若 $a, b \in \mathbb{R^+}$,求由方程为 $\dfrac{x^2_1}{a^2} + \dfrac{x^2_2}{b^2} = 1$ 的椭圆为边界的区域 $E$ 的面积\footnote{原题摘自《线性代数及其应用》(第三版)P183.}.
\end{sample}

\begin{solution}
    断言 $E$ 是单位圆盘 $D$ 在线性变换 $T$ 下的像. 这里 $T$ 由矩阵 $A = \begin{bmatrix}
        a & 0\\
        0 & b
    \end{bmatrix}$
    确定,这是因为若 $\mathbf{u} = \begin{bmatrix}
        u_1\\
        u_2
    \end{bmatrix}$,$\mathbf{x} = \begin{bmatrix}
        x_1\\
        x_2
    \end{bmatrix}$,且 $\mathbf{x} = A \mathbf{u}$,则
    \[
        u_1 = \frac{x_1}{a} , u_2 = \frac{x_2}{b}
    \]
    \noindent
    从而得 $\mathbf{u}$ 在此单位圆内,即满足 $u^2_1 + u^2_2 \leq 1$,
    当且仅当 $\mathbf{x}$ 在 $E$ 内,即满足 $(x_1/a)^2 + (x_2/b)^2 \leq 1$. 进而
    \noindent
    \begin{align*}  
        \{ \text{椭圆的面积} \} &= \{ T(D) \text{的面积} \}\\
        &= | \det A | \cdot \{ D \text{的面积} \}\\
        &= a \cdot b \cdot \pi \cdot (1)^2\\
        &= \pi a b
    \end{align*}
\end{solution}

\end{document}